\documentclass[11pt]{article}

\usepackage{fullpage}

\begin{document}

\title{ARM Checkpoint... }
\author{TODO}

\maketitle

\section{Group Organisation}
Pavan:

> Implemented emulator.c.
> Set up the basic framework and the skeleton for the project and also the
different structures required for the emulator such as the state struct(which handles the registers and memory)
and the cycle struct(which handles the pipeline execution).
> Implemented the execution flow of an instruction
(the process of checking the cond bits and differentiating the instruction type and passing it to the appropriate execution phase).
> Implemented the data_transfer_methods.c which handles all the methods and execution of a data transfer instruction.
> Implemented fetch,decode,execute pipeline in utils.h.

Hakeem:

> Implemented various utility functions in utils.c
> Implemented data_process_methods.c which handles the data processing instructions

Hamza:

> Implemented various utility functions in utils.c
> Implemented multiply_methods.c which handles the multiply instructions and branch_methods.c which handles the branch instructions



\section{Implementation Strategies}

we have the emulator more or less finished. Need to carry out some testing.
//Write about about how we organised and coordinating the work.
//Structure of the emulator

We structured the whole Emulator part as follows:

>emulator.c
This file initialise an instance of a state and cycle structures and initialises all variables in the state and cycle to 0.
It then reads each 32 bit instruction from the binary file and loads it into the state.data_mem block of memory. And calls the "start"
method in the utils with the pointers of state and cycle the perameters which triggers the pipeline execution phase.
>utils.c
This file is just a collection of useful functions required by the emulator and also contains the initialisation and data structures.
>utils.h
A headder file which defines the structures used for the emulator and also the signatures for utils.c.
>data_transfer_methods.c
this file contains all the methods required for the execution of Data transfer methods.
>data_transfer.h
A headder file which is used to specify the signature of each method in data_transfer_methods.c.
>branch_methods.c

>branch.h

One of the problems I had initially was actually deciding on how to differentiate between each type of instruction. There
was no obvious clear pattern for differentiating between the Multiply instructions and the Data Processing Instructions.
I think most of the methods in the utils.c are very general for example the bitsCheck method which only returns a 1 if two
uin32_t values are the same. Thus most of the methods in utils.c will be useful when implementing the assembler.

\end{document}
